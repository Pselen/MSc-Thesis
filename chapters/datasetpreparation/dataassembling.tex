\section{Data Assembling}

Our idea was to represent chemicals and proteins better. Therefore, we compiled chemical and proteins related data from several online databases. And the challenging task was to assemble these compiled large data. For that purpose we analyze the available information in above mentioned databases and and map related information using common data. 

\subsection{Chemical Related Information}
DrugBank, PubChem, and ChEMBL databases are the main resources of chemicals and we mainly focus on them. As an initial step we compile 14.350 drugs from DrugBank and retrieve their DrugBank IDs and InChi (International Chemical Identifier) Keys. Using InChI keys, we map the data in DrugBank to PubChem and ChEMBL databases and able to get the information about 10.935 distinct drugs. With 10.935 drugs, we extract 2.196.820 drug-drug relation information from DrugBank database. Using the PubChem CID (Compound ID number) information available at PubChem database, we map the PubChem to SIDER and CTD databases. From SIDER database, we extract 5452 distinct side effects and 115.871 drug-side effect association information for 1003 drugs. From CTD database, we extract 7086 distinct diseases and 995.654 drug-disease association information for 3387 drugs. Finally, using the InChI keys, we map DrugBank to ChEMBL and compile SMILES (Simplified molecular-input line-entry system) representations of 10.935 drugs. 

\subsection{Protein Related Information}
UniProt and STRING databases are main resources used in this thesis. We compile 505.250 proteins from the UniProt database and more specifically we compile 202.160 proteins belonging to the Homo sapiens, as well as their amino acid sequences. Using the UniProt ID from UniProt database, we map 18.876 proteins to STRING database, and extract 183.746 protein-protein interaction information. Finally, using the UniProt ID and Entrez Gene ID we map UniProt database to CTD and extract 32.495 protein-disease association information for 32.169 proteins and 126 distinct diseases.