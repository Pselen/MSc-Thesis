\chapter{RESULTS}

\section{Evaluation}
Representation vectors for chemicals and proteins obtained using the aforementioned models and graphs, and then these vectors evaluated in the drug-target affinity task with the DeepDTA model using the BDB dataset. The DeepDTA model represents proteins using amino acid sequences and chemicals using characters of the SMILES notations. In this study, the DeepDTA model has been updated, as shown in Figure 3.3, to take as input the representation vectors containing additional information in the generated graphs. The performance of the model is measured by the Concordance Index (CI),  Mean Squared Error (MSE), Root Mean Squared Error (RMSE), and R2 metrics. 

% explain metrics with formula

\section{Experimental Setup}

As the training and test folds we use the same setup used in DeepDTA \cite{ozturk2018deepdta}. We train each model 5 times with the training set folds, then measure the performance on each test set, and report the average results on BDB. BDB dataset contains 5 different training sets and corresponding 4 different test sets as warm, cold ligand, cold protein and cold. The cold test sets contain data that was not used during the training of the model. We compute CI, MSE, RMSE, and $R^2$ scores of each model and report with the standard deviation.


\section{Graph Creation}

\subsection{Homogeneous Graphs}
First, we generate homogeneous graphs with only one node and edge type:
\begin{enumerate}
    \item \textbf{Model (1)}: A DDI (Drug-Drug Interaction) graph is created, the nodes of which are formed by all drugs (D) and the edges by interactions (D-D) between these drugs.   
    \item \textbf{Model (2)}: A DDI graph is created similar to Model (1), this time with the knowledge of initial embeddings of ChemBERTa model.
    \item \textbf{Model (3)}: A PPI (Protein-Protein Interaction) graph is created, the nodes of which are formed by all proteins (P) and the edges by interactions (P-P) between these proteins.   
    \item \textbf{Model (4)}: A PPI graph is created similar to Model (3), this time with the knowledge of initial embeddings of ProtBERT model.
    \item \textbf{Model (9)}: A DDS (Drug-Drug Similarity) graph is created, the nodes of which are formed by all drugs (D) and the edges by Jaccard similarity between these drugs.   
    \item \textbf{Model (11)}: A PPS (Protein-Protein Similarity) graph is created, the nodes of which are formed by all proteins belonging to the human species (P) and the edges formed by the Jaccard similarities (P-P) between the amino acid sequences of these proteins. While calculating the similarities we tokenized each amino acid sequence using the glossary generated by BPE. Then we calculate the Paired Jaccard similarities of protein tokens. Considering the dataset, a threshold value is determined to cover at least 11\% of the data. Accordingly, protein pairs with similarity values greater than 9 determined to be similar to each other.
\end{enumerate}
\begin{table}
\centering
\caption{CI and R$^2$ scores of DDI models on test sets of BDB.}
\begin{tabular}{|l|l|c|c|} 
\hline
\textbf{Model} & \textbf{Test~Set} & \textbf{CI} & \textbf{R\textsuperscript{2}} \\ 
\hline
DeepDTA & \multirow{3}{*}{Warm} & \textbf{0.888 (0.009)} & \textbf{0.781 (0.028)} \\ 
\cline{1-1}\cline{3-4}
Model (1) &  & 0.884 (0.007) & 0.771 (0.018) \\ 
\cline{1-1}\cline{3-4}
Model (2) &  & 0.888 (0.011) & 0.777 (0.015) \\ 
\hline
DeepDTA & \multirow{3}{*}{\begin{tabular}[c]{@{}l@{}}Cold\\Ligand\end{tabular}} & \textbf{0.687 (0.096)} & \textbf{0.039 (0.243)} \\ 
\cline{1-1}\cline{3-4}
Model (1) &  & 0.684 (0.053) & -0.056 (0.257) \\ 
\cline{1-1}\cline{3-4}
Model (2) &  & 0.671 (0.070) & -0.104 (0.245) \\ 
\hline
DeepDTA & \multirow{3}{*}{\begin{tabular}[c]{@{}l@{}}Cold\\Protein\end{tabular}} & 0.759 (0.006) & 0.315 (0.049) \\ 
\cline{1-1}\cline{3-4}
Model (1) &  & \textbf{0.770 (0.019)} & \textbf{0.347 (0.087)} \\ 
\cline{1-1}\cline{3-4}
Model (2) &  & 0.769 (0.014) & 0.317 (0.077) \\ 
\hline
DeepDTA & \multirow{3}{*}{Cold} & 0.554 (0.047) & \textbf{-0.154 (0.164)} \\ 
\cline{1-1}\cline{3-4}
Model (1) &  & \textbf{0.582 (0.043)} & -0.302 (0.279) \\ 
\cline{1-1}\cline{3-4}
Model (2) &  & 0.554 (0.067) & -0.346 (0.305) \\
\hline
\end{tabular}
\label{tab:ddi_ci_r2}
\end{table}
\begin{table}
\centering
\caption{MSE and RMSE scores of DDI models on test sets of BDB.}
\begin{tabular}{|l|l|c|c|} 
\hline
\textbf{Model} & \textbf{Test~Set} & \textbf{MSE} & \textbf{RMSE} \\ 
\hline
DeepDTA & \multirow{3}{*}{Warm} & \textbf{0.288 (0.021)} & \textbf{0.536 (0.012)} \\ 
\cline{1-1}\cline{3-4}
Model (1) &  & 0.301 (0.015) & 0.548 (0.014) \\ 
\cline{1-1}\cline{3-4}
Model (2) &  & 0.294 (0.022) & 0.542 (0.020) \\ 
\hline
DeepDTA & \multirow{3}{*}{\begin{tabular}[c]{@{}l@{}}Cold\\Ligand\end{tabular}} & 1.448 (0.939) & \textbf{1.152 (0.348)} \\ 
\cline{1-1}\cline{3-4}
Model (1) &  & \textbf{1.421 (0.461)} & 1.176 (0.194) \\ 
\cline{1-1}\cline{3-4}
Model (2) &  & 1.500 (0.542) & 1.207 (0.211) \\ 
\hline
DeepDTA & \multirow{3}{*}{\begin{tabular}[c]{@{}l@{}}Cold\\Protein\end{tabular}} & 1.085 (0.146) & 1.040 (0.146) \\ 
\cline{1-1}\cline{3-4}
Model (1) &  & \textbf{1.033 (0.173)} & \textbf{1.013 (0.086)} \\ 
\cline{1-1}\cline{3-4}
Model (2) &  & 1.080 (0.162) & 1.036 (0.077) \\ 
\hline
DeepDTA & \multirow{3}{*}{Cold} & 2.007 (1.223) & \textbf{1.356 (0.410)} \\ 
\cline{1-1}\cline{3-4}
Model (1) &  & \textbf{1.985 (0.657)} & 1.390 (0.231) \\ 
\cline{1-1}\cline{3-4}
Model (2) &  & 2.105 (0.869) & 1.422 (0.290) \\
\hline
\end{tabular}
\label{tab:ddi_mse_rmse}
\end{table}

\begin{table}
\centering
\caption{CI and R$^2$ scores of DDS models on test sets of BDB.}
\label{tab:ddi_ci_r2}
\begin{tabular}{|l|l|c|c|} 
\hline
\begin{tabular}[c]{@{}l@{}}\\\textbf{Model}\end{tabular} & \textbf{Test~Set} & \textbf{CI} & \textbf{R\textsuperscript{2}} \\ 
\hline
DeepDTA & \multirow{3}{*}{Warm} & \textbf{0.888 (0.009)} & \textbf{0.781 (0.028)} \\ 
\cline{1-1}\cline{3-4}
Model (3) &  & 0.883 (0.006) & 0.766 (0.019) \\ 
\cline{1-1}\cline{3-4}
Model (4) &  & 0.887 (0.008) & 0.769 (0.012) \\ 
\hline
DeepDTA & \multirow{3}{*}{\begin{tabular}[c]{@{}l@{}}Cold\\Ligand\end{tabular}} & \textbf{0.687 (0.096)} & \textbf{0.039 (0.243)} \\ 
\cline{1-1}\cline{3-4}
Model (3) &  & 0.635 (0.062) & -0.362 (0.323) \\ 
\cline{1-1}\cline{3-4}
Model (4) &  & 0.681 (0.061) & 0.005 (0.185) \\ 
\hline
DeepDTA & \multirow{3}{*}{\begin{tabular}[c]{@{}l@{}}Cold\\Protein\end{tabular}} & 0.759 (0.006) & 0.315 (0.049) \\ 
\cline{1-1}\cline{3-4}
Model (3) &  & 0.764 (0.027) & \textbf{0.345 (0.060)} \\ 
\cline{1-1}\cline{3-4}
Model (4) &  & \textbf{0.773 (0.019)} & 0.332 (0.077) \\ 
\hline
DeepDTA & \multirow{3}{*}{Cold} & \textbf{0.554 (0.047) } & \textbf{-0.154 (0.164)} \\ 
\cline{1-1}\cline{3-4}
Model (3) &  & 0.505 (0.032) & -0.524 (0.378) \\ 
\cline{1-1}\cline{3-4}
Model (4) &  & 0.545 (0.051) & -0.224 (0.174) \\
\hline
\end{tabular}
\label{tab:dds_ci_r2}
\end{table}


\begin{table}
\centering
\caption{MSE and RMSE scores of DDS models on test sets of BDB.}
\begin{tabular}{|l|l|c|c|} 
\hline
\textbf{Model} & \textbf{Test~Set} & \textbf{MSE} & \textbf{RMSE} \\ 
\hline
DeepDTA & \multirow{3}{*}{Warm} & \textbf{0.288 (0.021)} & \textbf{0.536 (0.012)} \\ 
\cline{1-1}\cline{3-4}
Model (3) &  & 0.308 (0.018) & 0.555 (0.016) \\ 
\cline{1-1}\cline{3-4}
Model (4) &  & 0.304 (0.016) & 0.551 (0.014) \\ 
\hline
DeepDTA & \multirow{3}{*}{\begin{tabular}[c]{@{}l@{}}Cold\\Ligand\end{tabular}} & 1.448 (0.939) & 1.152 (0.348) \\ 
\cline{1-1}\cline{3-4}
Model (3) &  & 1.903 (0.865) & 1.348 (0.292) \\ 
\cline{1-1}\cline{3-4}
Model (4) &  & \textbf{1.360 (0.501)} & \textbf{1.149 (0.199)} \\ 
\hline
DeepDTA & \multirow{3}{*}{\begin{tabular}[c]{@{}l@{}}Cold\\Protein\end{tabular}} & 1.085 (0.146) & 1.040 (0.146) \\ 
\cline{1-1}\cline{3-4}
Model (3) &  & \textbf{1.037 (0.150)} & \textbf{1.016 (0.073)} \\ 
\cline{1-1}\cline{3-4}
Model (4) &  & 1.056 (0.150) & 1.025 (0.074) \\ 
\hline
DeepDTA & \multirow{3}{*}{Cold} & 2.007 (1.223) & \textbf{1.356 (0.410)} \\ 
\cline{1-1}\cline{3-4}
Model (3) &  & 2.380 (1.050) & 1.510 (0.316) \\ 
\cline{1-1}\cline{3-4}
Model (4) &  & \textbf{1.957 (0.908)} & 1.367 (0.299) \\
\hline
\end{tabular}
\label{tab:dds_mse_rmse}
\end{table}



\begin{table}
\centering
\caption{CI and R$^2$ scores of PPI models on test sets of BDB.}
\label{tab:ddi_ci_r2}
\begin{tabular}{|l|l|c|c|} 
\hline
\begin{tabular}[c]{@{}l@{}}\textbf{Model}\end{tabular} & \textbf{Test~Set} & \textbf{CI} & \textbf{R\textsuperscript{2}} \\ 
\hline
DeepDTA & \multirow{3}{*}{Warm} & \textbf{0.888 (0.009)} & \textbf{0.781 (0.028)} \\ 
\cline{1-1}\cline{3-4}
Model (5) &  & 0.884 (0.007) & 0.770 (0.009) \\ 
\cline{1-1}\cline{3-4}
Model (6) &  & \textbf{0.888 (0.007)} & 0.779 (0.019) \\ 
\hline
DeepDTA & \multirow{3}{*}{\begin{tabular}[c]{@{}l@{}}Cold\\Ligand\end{tabular}} & 0.687 (0.096) & 0.039 (0.243) \\ 
\cline{1-1}\cline{3-4}
Model (5) &  & 0.681 (0.108) & -0.029 (0.304) \\ 
\cline{1-1}\cline{3-4}
Model (6) &  & \textbf{0.695 (0.040)} & \textbf{0.043 (0.054)} \\ 
\hline
DeepDTA & \multirow{3}{*}{\begin{tabular}[c]{@{}l@{}}Cold\\Protein\end{tabular}} & \textbf{0.759 (0.006)} & \textbf{0.315 (0.049)} \\ 
\cline{1-1}\cline{3-4}
Model (5) &  & 0.758 (0.020) & 0.333 (0.096) \\ 
\cline{1-1}\cline{3-4}
Model (6) &  & \textbf{0.759 (0.016)} & \textbf{0.315 (0.080)} \\ 
\hline
DeepDTA & \multirow{3}{*}{Cold} & 0.554 (0.047) & \textbf{-0.154 (0.164)} \\ 
\cline{1-1}\cline{3-4}
Model (5) &  & 0.564 (0.052) & -0.162 (0.175) \\ 
\cline{1-1}\cline{3-4}
Model (6) &  & \textbf{0.584 (0.048)} & -0.156 (0.112) \\
\hline
\end{tabular}
\label{tab:ppi_ci_r2}
\end{table}
\begin{table}
\centering
\caption{MSE and RMSE scores of PPI models on test sets of BDB.}
\label{tab:ddi_ci_r2}
\begin{tabular}{|l|l|c|c|} 
\hline
\begin{tabular}[c]{@{}l@{}}\textbf{Model}\end{tabular} & \textbf{Test~Set} & \textbf{MSE} & \textbf{RMSE} \\ 
\hline
DeepDTA & \multirow{3}{*}{Warm} & \textbf{0.288 (0.021)} & 0.304 (0.015) \\ 
\cline{1-1}\cline{3-4}
Model (5) &  & 0.304 (0.015) & 0.291 (0.019) \\ 
\cline{1-1}\cline{3-4}
Model (6) &  & 0.291 (0.019) & 0.539 (0.017) \\ 
\hline
DeepDTA & \multirow{3}{*}{\begin{tabular}[c]{@{}l@{}}Cold\\Ligand\end{tabular}} & 1.448 (0.939) & 1.152 (0.348) \\ 
\cline{1-1}\cline{3-4}
Model (5) &  & 1.539 (1.044) & 1.185 (0.366) \\ 
\cline{1-1}\cline{3-4}
Model (6) &  & \textbf{1.316 (0.433)} & \textbf{1.133 (0.179)} \\ 
\hline
DeepDTA & \multirow{3}{*}{\begin{tabular}[c]{@{}l@{}}Cold\\Protein\end{tabular}} & 1.085 (0.146) & 1.040 (0.146) \\ 
\cline{1-1}\cline{3-4}
Model (5) &  & 1.053 (0.175) & 1.023 (0.083) \\ 
\cline{1-1}\cline{3-4}
Model (6) &  & \textbf{1.081 (0.152)} & \textbf{1.037 (0.072)} \\ 
\hline
DeepDTA & \multirow{3}{*}{Cold} & 2.007 (1.223) & 1.356 (0.410) \\ 
\cline{1-1}\cline{3-4}
Model (5) &  & 1.936 (1.101) & 1.344 (0.360) \\ 
\cline{1-1}\cline{3-4}
Model (6) &  & \textbf{1.824 (0.731)} & \textbf{1.326 (0.256)} \\
\hline
\end{tabular}
\label{tab:ppi_mse_rmse}
\end{table}

\begin{table}
\centering
\caption{CI and R$^2$ scores of PPS models on test sets of BDB.}
\label{tab:ddi_ci_r2}
\begin{tabular}{|l|l|c|c|} 
\hline
\textbf{Model} & \textbf{Test~Set} & \textbf{CI} & \textbf{R\textsuperscript{2}} \\ 
\hline
DeepDTA & \multirow{3}{*}{Warm} & \textbf{0.888 (0.009)} & \textbf{0.781 (0.028)} \\ 
\cline{1-1}\cline{3-4}
Model (7) &  & 0.885 (0.004) & 0.761 (0.012) \\ 
\cline{1-1}\cline{3-4}
Model (8) &  & 0.887 (0.005) & 0.773 (0.013) \\ 
\hline
DeepDTA & \multirow{3}{*}{\begin{tabular}[c]{@{}l@{}}Cold\\Ligand\end{tabular}} & \textbf{0.687 (0.096)} & \textbf{0.039 (0.243)} \\ 
\cline{1-1}\cline{3-4}
Model (7) &  & 0.677 (0.080) & -0.142 (0.138) \\ 
\cline{1-1}\cline{3-4}
Model (8) &  & 0.662 (0.076) & -0.177 (0.267) \\ 
\hline
DeepDTA & \multirow{3}{*}{\begin{tabular}[c]{@{}l@{}}Cold\\Protein\end{tabular}} & 0.759 (0.006) & 0.315 (0.049) \\ 
\cline{1-1}\cline{3-4}
Model (7) &  & 0.768 (0.015) & 0.333 (0.054) \\ 
\cline{1-1}\cline{3-4}
Model (8) &  & \textbf{0.775 (0.012)} & \textbf{0.358 (0.090)} \\ 
\hline
DeepDTA & \multirow{3}{*}{Cold} & 0.554 (0.047) & \textbf{-0.154 (0.164)} \\ 
\cline{1-1}\cline{3-4}
Model (7) &  & 0.581 (0.033) & -0.316 (0.198) \\ 
\cline{1-1}\cline{3-4}
Model (8) &  & \textbf{0.558 (0.047)} & -0.307 (0.193) \\
\hline
\end{tabular}
\label{tab:pps_ci_r2}
\end{table}
% \usepackage{multirow}


\begin{table}
\centering
\caption{MSE and RMSE scores of PPS models on test sets of BDB.}
\label{tab:ddi_ci_r2}
\begin{tabular}{|l|l|c|c|} 
\hline
\begin{tabular}[c]{@{}l@{}}\\\textbf{Model}\end{tabular} & \textbf{Test~Set} & \textbf{MSE} & \textbf{RMSE} \\ 
\hline
DeepDTA & \multirow{3}{*}{Warm} & \textbf{0.288 (0.021)} & \textbf{0.536 (0.012)} \\ 
\cline{1-1}\cline{3-4}
Model (7) &  & 0.316 (0.025) & 0.562 (0.022) \\ 
\cline{1-1}\cline{3-4}
Model (8) &  & 0.300 (0.012) & 0.547 (0.011) \\ 
\hline
DeepDTA & \multirow{3}{*}{\begin{tabular}[c]{@{}l@{}}Cold\\Ligand\end{tabular}} & \textbf{1.448 (0.939)} & \textbf{1.152 (0.348)} \\ 
\cline{1-1}\cline{3-4}
Model (7) &  & 1.589 (0.602) & 1.240 (0.229) \\ 
\cline{1-1}\cline{3-4}
Model (8) &  & 1.715 (0.985) & 1.265 (0.337) \\ 
\hline
DeepDTA & \multirow{3}{*}{\begin{tabular}[c]{@{}l@{}}Cold\\Protein\end{tabular}} & 1.085 (0.146) & 1.040 (0.146) \\ 
\cline{1-1}\cline{3-4}
Model (7) &  & 1.056 (0.134) & 1.025 (0.066) \\ 
\cline{1-1}\cline{3-4}
Model (8) &  & \textbf{1.017 (0.184)} & \textbf{1.005 (0.089)} \\ 
\hline
DeepDTA & \multirow{3}{*}{Cold} & \textbf{2.007 (1.223)} & \textbf{1.356 (0.410)} \\ 
\cline{1-1}\cline{3-4}
Model (7) &  & 2.074 (0.873) & 1.412 (0.282) \\ 
\cline{1-1}\cline{3-4}
Model (8) &  & 2.164 (1.221) & 1.423 (0.372) \\
\hline
\end{tabular}
\label{tab:ppse_mse_rmse}
\end{table}

\subsection{Heterogeneous Graphs}
Second, we generate heterogeneous graphs with several node and edge types:
\begin{enumerate}
    \item \textbf{Model (5)} A DSA (Drug-Side Effect Association) graph is created, the nodes of which are formed by all drugs (D), side effects (S), and the edges by association (D-S) between these drugs and side effects.  
    \item \textbf{Model (6)} A DSA graph is created similar to Model (5), this time with the knowledge of initial embeddings of ChemBERTa and BioBert models for drugs and side effects respectively. 
    \item \textbf{Model (7)} A DDiPA (Drug-Disease-Protein Association) graph is created, the nodes of which are formed by all drugs (D), diseases (Di), proteins (P) and the edges by association (D-Di-P) between these drugs, proteins, and diseases.  
    \item \textbf{Model (8)} A DDiPA graph is created similar to Model (7), this time with the knowledge of initial embeddings of ChemBERTa, BioBert, and ProtBERT models for drugs, diseases and proteins respectively. 
\end{enumerate}



\begin{table}
\centering
\caption{CI and R$^2$ scores of DDiA models on test sets of BDB.}
\label{tab:ddi_ci_r2}
\begin{tabular}{|l|l|c|c|} 
\hline
\textbf{Model} & \textbf{Test~Set} & \textbf{CI} & \textbf{R\textsuperscript{2}} \\ 
\hline
DeepDTA & \multirow{3}{*}{Warm} & \textbf{0.888 (0.009)} & \textbf{0.781 (0.028)} \\ 
\cline{1-1}\cline{3-4}
Model (9) &  & 0.885 (0.009) & 0.773 (0.015) \\ 
\cline{1-1}\cline{3-4}
Model (10) &  & 0.885 (0.006) & 0.773 (0.005) \\ 
\hline
DeepDTA & \multirow{3}{*}{\begin{tabular}[c]{@{}l@{}}Cold\\Ligand\end{tabular}} & 0.687 (0.096) & 0.039 (0.243) \\ 
\cline{1-1}\cline{3-4}
Model (9) &  & \textbf{0.704 (0.039)} & -0.082 (0.377) \\ 
\cline{1-1}\cline{3-4}
Model (10) &  & 0.699 (0.045) & \textbf{0.042 (0.120)} \\ 
\hline
DeepDTA & \multirow{3}{*}{\begin{tabular}[c]{@{}l@{}}Cold\\Protein\end{tabular}} & 0.759 (0.006) & 0.315 (0.049) \\ 
\cline{1-1}\cline{3-4}
Model (9) &  & 0.774 (0.010) & 0.335 (0.077) \\ 
\cline{1-1}\cline{3-4}
Model (10) &  & \textbf{0.777 (0.020)} & \textbf{0.348 (0.067)} \\ 
\hline
DeepDTA & \multirow{3}{*}{Cold} & 0.554 (0.047) & -0.154 (0.164) \\ 
\cline{1-1}\cline{3-4}
Model (9) &  & 0.616 (0.077) & -0.183 (0.363) \\ 
\cline{1-1}\cline{3-4}
Model (10) &  & \textbf{0.618 (0.042)} & \textbf{-0.084 (0.139)} \\
\hline
\end{tabular}
\label{tab:ddia_ci_r2}
\end{table}
% \usepackage{multirow}


\begin{table}
\centering
\caption{MSE and RMSE scores of DDiA models on test sets of BDB.}
\label{tab:ddi_ci_r2}
\begin{tabular}{|l|l|c|c|} 
\hline
\textbf{Model} & \textbf{Test~Set} & \textbf{MSE} & \textbf{RMSE} \\ 
\hline
DeepDTA & \multirow{3}{*}{Warm} & \textbf{0.288 (0.021)} & \textbf{0.536 (0.012)} \\ 
\cline{1-1}\cline{3-4}
Model (9) &  & 0.299 (0.021) & 0.547 (0.019) \\ 
\cline{1-1}\cline{3-4}
Model (10) &  & 0.299 (0.014) & 0.547 (0.013) \\ 
\hline
DeepDTA & \multirow{3}{*}{\begin{tabular}[c]{@{}l@{}}Cold\\Ligand\end{tabular}} & 1.448 (0.939) & 1.152 (0.348) \\ 
\cline{1-1}\cline{3-4}
Model (9) &  & 1.416 (0.432) & 1.175 (0.189) \\ 
\cline{1-1}\cline{3-4}
Model (10) &  & \textbf{1.304 (0.415)} & \textbf{1.129 (0.171)} \\ 
\hline
DeepDTA & \multirow{3}{*}{\begin{tabular}[c]{@{}l@{}}Cold\\Protein\end{tabular}} & 1.085 (0.146) & 1.040 (0.146) \\ 
\cline{1-1}\cline{3-4}
Model (9) &  & 1.053 (0.167) & 1.023 (0.081) \\ 
\cline{1-1}\cline{3-4}
Model (10) &  & \textbf{1.032 (0.154)} & \textbf{1.013 (0.075)} \\ 
\hline
DeepDTA & \multirow{3}{*}{Cold} & 2.007 (1.223) & 1.356 (0.410) \\ 
\cline{1-1}\cline{3-4}
Model (9) &  & 1.804 (0.715) & 1.317 (0.261) \\ 
\cline{1-1}\cline{3-4}
Model (10) &  & \textbf{1.715 (0.724)} & \textbf{1.284 (0.259)} \\
\hline
\end{tabular}
\label{tab:ddia_mse_rmse}
\end{table}