\begin{ozet}
İlaç-hedef bağlılık ilgisi tahmini, bilgisayar destekli ilaç tasarımında, ilaç geliştir-
\\me sürecini hızlandırmaya ve çok sayıda bulunan yanlış pozitif oranlarının neden olduğu deneysel doğrulama maliyetlerini düşürmeye yardımcı olabilecek kritik bir aşamadır. Bu nedenle, ilaç-hedef bağlılık ilgisi değerlerini tahmin etmek için bilgisayar ortamında hesaplama algoritmaları geliştirmek ilgi çekici bir araştırma alanı haline gelmiştir. Güncel çalışmalar bu görev için, kolayca bulunabilen biyomolekül dizilerini ve ilaçlarla ve hedeflerle alakalı bilgilerle zenginleştirilmiş heterojen ağları kullanan modeller de dahil olmak üzere, makine öğrenimi yaklaşımlarını kullanır. Bu tezde, hem metin tabanlı hem de ağ tabanlı yaklaşımlardan yararlanan ve ilaç-hedef bağlılık ilgisi değerlerini tahmin eden ilk çalışma olan WideDeepDTA'yı sunuyoruz. WideDeepDTA içerisinde birden fazla biyolojik varlık türü, bu varlıklar arasındaki ilişkiler ve biyomoleküler dil için önceden eğitilmiş dil modellerini içeren homojen ve heterojen ağları barındırır. Tüm bunlar göz önüne alındığında, WideDeepDTA önce ağlarda bulunan tüm düğümler için bir vektör gösterim öğrenme yöntemi olan Metapath2Vec'i kullanarak ilaçların ve hedeflerin düşük boyutlu vektör temsillerini öğrenir. Ardından, öğrenilen temsillere dayanarak ilaç-hedef bağlılık ilgisi değerlerini tahmin eder. WideDeepDTA, BDB veri kümesindeki en başarılı yöntemlerden biri olan DeepDTA'ya kıyasla ilaç-hedef bağlılık ilgisi tahmini görevinde uyumluluk indeksi ve ortalama kare hata başarı metriklerinde iyileşme göstererek zengin temsiller oluşturmayı başarmıştır. Yapılan deneyler, ilaçlar ve proteinler için önceden eğitilmiş dil modellerini heterojen ağlarla birlikte kullanmamın model performansını geliştirdiği göstermektedir. Ayrıca sonuçlar, metin tabanlı temsillerden elde edilen bilgilerle heterojen ağlar güçlendirildiğinde model performansının arttığını göstermektedir.
\end{ozet}