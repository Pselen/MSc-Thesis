\chapter{RELATED WORK}
\label{related_work}

Traditional computational methods used in drug discovery are based on four strategies; ligand similarity-based \cite{keiser2007relating}, molecular docking/structure-based \cite{morris2009autodock4,donald2011algorithms}, deep learning-based \cite{wan2018neodti, luo2017network}, and network-based approaches \cite{luo2017network, zheng2013collaborative, chen2012drug, wang2014drug}. The performance of the ligand similarity-based approaches is often low when a target has a few known binding ligands. Likewise, the limited availability of 3D structures of target proteins limits the molecular docking performance. In the last decade, some effort has been devoted to developing machine learning-based approaches for drug target affinity (DTA) predictions through computational techniques. Binding affinity provides information on the strength of the interaction between a drug-target pair and the increase in available affinity data in online databases has led to the use of advanced learning techniques such as deep learning architectures in predicting binding affinities \cite{chan2016large, tian2016boosting, hamanaka2017cgbvs}.

Protein-ligand scoring is used to approximately predict the binding affinity between two molecules, and it is frequently used after virtual screening and docking campaigns \cite{ragoza2017protein}. One of the successful alternative machine learning methods to scoring functions is Random Forest algorithm \cite{ballester2010machine, shar2016pred}. However, it fails in virtual screening and docking tests due to oversimplification of the protein-ligand complex descriptions \cite{gabel2014beware}. The continuously expanding amount of protein-ligand binding data enables the use of deep learning methods in scoring. For instance, 3D structures of protein-ligand complexes are commonly used with Convolutional Neural Networks (CNNs) \cite{gomes2017atomic, ragoza2017protein, wallach2015atomnet}, however, its success is limited to known protein-ligand complex structures. Kronecker Regularized Least Squares (KronRLS) algorithm is also used in scoring \cite{pahikkala2014toward}. It utilizes only 2D based compound similarity-based representations of the drugs and Smith-Waterman similarity representation of the targets. Another approach proposed to predict the binding affinity scores is SimBoost method \cite{he2017simboost}. It uses a gradient boosting machine with the extracted features from interactions between drug-target pairs and their similarity-based information. 

Several types of deep learning frameworks have been adopted in DTA prediction task. DeepDTA \cite{ozturk2018deepdta} and WideDTA \cite{ozturk2019widedta} approaches are proposed to predict the binding affinities of protein-ligand interactions. Both methods utilize deep learning models that use only 1D representations of proteins and ligands. As the 1D representation, both of the studies use SMILES (Simplified Molecular Input Line Entry System) representations of the compounds rather than complex external features or 3D-structures of the binding complexes. DeepDTA learns high dimensional features from full-length sequences of the proteins and ligands. It uses two CNNs to learn the representations of drugs and proteins. Then, the concatenated representations of drugs and proteins are fed into the a multi-layer perceptron (MLP). Yet, it fails to capture the biologically important short subsequences. WideDTA overcomes this problem by integrating different kinds of text-based information such as protein sequence, ligand SMILES, protein domains and motifs, and maximum common substructure words to provide better representation and to predict binding affinity. To do that, WideDTA employs four CNNs and learn the representations of drugs and proteins. Similarly, uses the MLP with the concatenated representations. Lee \textit{et al.} \cite{lee2019deepconv} also utilizes CNNs on the protein sequences. On the other hand, they use 2D structural images of chemicals and learn complex features from them using CNNs and produce DTA predictions.

Although the extensive experiments and remarkable performance in DTA prediction task, representing the drugs as strings cause loss of information due to structural information lies beneath the molecules. To address this problem, graph neural networks (GNNs) are employed and drugs are represented as graphs. Tsubaki \textit{et al.} \cite{tsubaki2019compound} propose to use CNNs and GNNs together to learn the representation of compound graphs and protein sequences. They demonstrate performance improvement on DTA task compared to the feature-based methods. GraphDTA \cite{nguyen2019graphdta} also suggests a new neural network architecture for drug-target affinity prediction task. Rather than using the 1D representation of SMILES, they convert SMILES representation into a molecular graph and employ a graph neural network (GNN) to learn a graph representation. Moreover, they encode and embed protein amino acid sequences and use CNN to create protein representations. Then, combine CNNs and GNNs to predict the binding affinity value. Another method, DGraphDTA \cite{jiang2020drug}, uses graphs to represent both compounds and proteins and with GNNs. Additionally, to address the interpretability, several models employ attention mechanism \cite{karimi2020explainable, chen2020transformercpi, agyemang2020multi, yang2021ml}. 

Rather than using only the known drug-target interaction (DTI) data in deep learning models, some other diverse information from heterogeneous data sources integrated into the systems, such as protein-protein interaction (PPI), drug-disease association, drug-side effect association as in the work of MSCMF \cite{zheng2013collaborative}, HNM \cite{wang2014drug}, DTINet \cite{luo2017network}, and NeoDTI \cite{wan2018neodti}. They employ networks that are able to capture the complex relation between different types of components such as drugs and proteins. These methods have improved the performance in DTI prediction task, yet they have some limitations to be addressed. In MSCMF \cite{zheng2013collaborative}, drug and protein similarity matrices are gathered from different data sources via a weighted averaging scheme in order to use in matrix factorization of a given DTI network. However, this data integration often causes data loss that will result in a suboptimal solution. DTINet \cite{luo2017network} is developed as a computational pipeline to predict novel DTI from a heterogeneous network. First, it learns low-dimensional feature representations of drugs and targets with an unsupervised manner, then it predicts new DTIs with inductive matrix completion (IMC) as in the work of Natarajan and Dhillon \cite{natarajan2014inductive}. Since DTINet handles the unsupervised feature learning procedure and the prediction task separately, it may cause non-optimal solutions. NeoDTI \cite{wan2018neodti} targets this problem and combines feature learning and classification into a single task, improving the accuracy. 

More recently, Zhao \textit{et al.} \cite{zhao2021identifying} propose a method that combines GNNs and deep neural networks (DNNs) for DTI prediction task. They build a drug-protein network using drug-drug interaction, protein-protein interaction, and drug-protein interaction networks in which a nodes represent drug and proteins and edges represent the link strength between them. Then, handle the DTI prediction problem as a node classification problem. Another network-based method EEG-DTI \cite{peng2021end} proposes an end-to-end heterogeneous graph representation learning-based framework to predict the interaction between drugs and targets using graph convolutional networks (GCNs). DTiGEMS$+$ \cite{thafar2020dtigems+} constructs a heterogeneous graph using the DTI graph with drug-drug similarity and target-target similarity graphs. It combines feature-based and similarity-based approaches to model the identification of drug-target pairs. After performing graph augmentation, it applies node2vec \cite{grover2016node2vec} for feature representation learning of drugs and targets and use them in a link prediction task. To improve the DTiGEMS$+$'s performance, DTi2Vec \cite{thafar2021dti2vec} is proposed in which representation learning and ensemble learning techniques are combined to identify the drug-target interactions. Unlike the previous work, it uses edge embeddings between drug-target node pair rather than the node embeddings.

Given the success of heterogeneous graphs in DTI prediction and text-based methods in DTA prediction, in this thesis we propose a method for DTA prediction that utilizes heterogeneous graphs. Moreover, in order to enrich the heterogeneity of the graph, we propose to add biomolecular language-based information obtained from the 1D structure. To the best of our knowledge, this is the first study that combines heterogeneous graphs and biomolecular language-based information for the DTA prediction task. 