\chapter{INTRODUCTION}
<<<<<<< HEAD
\label{chapter:intro}
\pagenumbering{arabic}
Drug design is a costly and time-consuming process that can be accomplished by discovering new candidate chemicals and evaluating these chemicals against various protein targets \cite{csermely2013structure}. The main goal of drug design is to provide a selective effect while minimizing the side effects by targeting only the disease-specific receptors and protecting the healthy cells \cite{hughes2011principles}. Today, rational designs that save time and cost in the pharmaceutical design are applied, and possible to develop drugs with selectively practical and fewer side effects \cite{huggins2012rational}. This rational discovery process often begins with developing a drug-active substance by selecting and improving ligands from a molecule library \cite{wen2015drug}. Proteins, DNA, RNA, and other small molecules can all interact with a drug. However, the size of the drug search space is enormous when we consider the existence of 100 million chemicals in the chemical database PubChem \cite{bolton2008pubchem}, the 16,526 drugs in the DrugBank \cite{law2013drugbank}, and over 189 million proteins in UniProt \cite{uniprot2021uniprot}.

Generating novel drug molecules considers several properties. For instance, a drug molecule should be synthesizable and should have target specificity, \textit{i.e.}, it should have binding affinity to the target protein of interest. On the other hand, it should have off-target selectivity so that its binding affinity is low to other targets. These properties make drug discovery cost over 3 billion USD, consume more time than a decade, and give a success rate of less than $10\%$ \cite{matthews2016omics}. Due to the expansive search space, high cost, and time consumption, the need for computational methods has emerged for this multi-stage, trial and error-based process \cite{yu2017computer}. 

Traditional computational drug design relies on simulations, heuristic search algorithms, and extensive domain knowledge. Furthermore, there is a great deal of interest in developing machine learning algorithms that can efficiently discover a large number of plausible and novel candidate drugs. Recently, deep learning approaches have gained attraction in the in silico drug design, increasing the available data and computing power of computers. 

As an initial step, studies try to elaborate on a drug's interacting targets, and computational methods try to determine the interacting and non-interacting drug and target pairs and use binary classification methods \cite{yamanishi2010drug, liu2016neighborhood, nascimento2016multiple, keum2017self, greenside2017prediction} for that purpose. However, the strength of the protein-ligand interactions, \textit{i.e.,} the binding affinity, is essential in the drug design pipeline since a strong interaction is the first step in finding a selective drug \cite{kawasaki2011finding}. Binary classification-based approaches provide information about a possible interaction between proteins and ligands; however, these methods cannot determine binding affinity. Therefore, the prediction of the binding affinity value still remains a challenge \cite{ozturk2018deepdta}. 

In order for a machine learning algorithm to interpret any type of input, it requires to get some effective representation, that is, vectorization \cite{mcculloch1943logical}. In the case of the computer-aided drug design, ligands, proteins, or any other types of biomolecule-related data need to be vectorized and represented numerically \cite{lecun2015deep}. Text-based or graph-based representation approaches are commonly employed to vectorize biomolecules. These representations are then used as training data in drug discovery studies in order to learn the relations between them or make some inference about their interactions \cite{osman2020graph,jiang2021could, jin2021embeddti,shatkay2015text}. Chemical or protein sequences or structures are some of the crucial data for drug-target interaction or the affinity prediction task; however, in online databases, there are many other types of available information related to chemicals and targets which are proven to affect the binding process \cite{kastritis2013binding, wang2020predicting}, such as the associated diseases of proteins, the side effects of the drug molecule, and the other interacting drugs or proteins. Therefore, rather than using only one type of information while learning the representation, studies show that integrating more information into the representations increases their ability to learn more features \cite{ling2017integrating, wang2017integrating, luo2017network, moon2021learning} of the data, resulting in richer representations. Inspired by the richer representations of biomolecules, this study integrates drug and protein sequences, the text-based similarities of biomolecule sequences, associated diseases, and side effects while learning the representation vectors of drugs and proteins using a heterogeneous network-based approach to the drug-target affinity prediction task. This is the first study that combines heterogeneous graphs and biomolecular language-based information for the drug-target affinity prediction task to the best of our knowledge. 

%We expect that using this existing related information will increase the quality of the representations and increase the success rate of drug design. Therefore, rather than using only the text-based distributional representation models of molecules, we also want to use network-based representation vectors in our model through heterogeneous graphs constructed using available related information. This thesis proposes a heterogeneous graph-based model that is enriched with biomolecular language-based information to predict the drug-target binding affinities.


% My overall comment on intro:
% It has some high-quality points like selectivity with high affinity (needs ref). They need to be polished and clarified. Currently, intro has repetitions and not progressing linearly. My advise is: prepare an outline for intro s.t. each item of the outline is a sentence or a phrase. Then expand each item as a paragraph. To see why would this be useful, do the inverse first, if you will. i.e. summarize each of the current paragraphs with one sentence. You'll see that it doesn't progress.

% Plus, one very important point: Intro needs many many many references. You are not an expert on drug design and can't write sentences with this precision in intro. For the computational part, there are thousands of possible refs. Use them. 
=======
\label{chapter:introduction}
\pagenumbering{arabic}
Drug design is an expensive and time-consuming process that can be carried out by testing the existing chemicals on different types of protein targets \cite{csermely2013structure}. Furthermore, drug design is a highly dynamic field due to the continuous evolution of target proteins and their different responses to the same drugs over time. During the process, several factors are considered such as protein-ligand binding affinity, bioactive conformation, pharmacokinetic parameters, metabolic stability, selectivity, toxicity, and synthesizability.

The main goal of drug design is to provide a selective effect while minimizing the side-effects by targeting only the disease-specific receptors and protecting the healthy cells. Today, rational designs that save time and cost in the pharmaceutical design are applied and it is possible to develop drugs with selectively effective and fewer side effects. This rational discovery process often begins with the development of a drug active substance, by selecting and improving ligands from a molecule library. However, the size of the drug search space is huge when we consider the existence of 97 million chemicals in the chemical database PubChem \cite{bolton2008pubchem} and the 16,526 drugs in the DrugBank \cite{law2013drugbank}. Due to the expansive search space, the need for computational methods has emerged for this multi-stage, trial and error based process. 

Computational methods aim to determine the interacting and non-interacting drug and target pairs, and binary classification methods have been commonly used \cite{yamanishi2010drug, liu2016neighborhood, nascimento2016multiple, keum2017self, greenside2017prediction}. Binary classification-based approaches provide information about a possible interaction between proteins and ligands, however, the strength of the protein-ligand interactions, namely the binding affinity, cannot be determined with these methods. Binding affinity is important  in the drug design pipeline since a strong interaction is the first step in finding a selective drug. However, prediction of the binding affinity value still remains a challenge \cite{ozturk2018deepdta}. In this thesis, we propose a graph-based model to predict the drug target binding affinities.
<<<<<<< Updated upstream
% Continue with the actual proposal
=======
% Continue with the actual proposal
>>>>>>> d09ebba9e602d1522ef9361417dc21d129009597
>>>>>>> Stashed changes
