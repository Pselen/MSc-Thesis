\chapter{INTRODUCTION}
\label{chapter:introduction}
\pagenumbering{arabic}
Drug design is an expensive and time-consuming process that can be carried out by testing the existing chemicals on different types of protein targets \cite{csermely2013structure}. Furthermore, drug design is a highly dynamic field due to the continuous evolution of target proteins and their different responses to the same drugs over time. During the process, several factors are considered such as protein-ligand binding affinity, bioactive conformation, pharmacokinetic parameters, metabolic stability, selectivity, toxicity, and synthesizability.

The main goal of drug design is to provide a selective effect while minimizing the side-effects by targeting only the disease-specific receptors and protecting the healthy cells. Today, rational designs that save time and cost in the pharmaceutical design are applied and it is possible to develop drugs with selectively effective and fewer side effects. This rational discovery process often begins with the development of a drug active substance, by selecting and improving ligands from a molecule library. However, the size of the drug search space is huge when we consider the existence of 97 million chemicals in the chemical database PubChem \cite{bolton2008pubchem} and the 16,526 drugs in the DrugBank \cite{law2013drugbank}. Due to the expansive search space, the need for computational methods has emerged for this multi-stage, trial and error based process. 

Computational methods aim to determine the interacting and non-interacting drug and target pairs, and binary classification methods have been commonly used \cite{yamanishi2010drug, liu2016neighborhood, nascimento2016multiple, keum2017self, greenside2017prediction}. Binary classification-based approaches provide information about a possible interaction between proteins and ligands, however, the strength of the protein-ligand interactions, namely the binding affinity, cannot be determined with these methods. Binding affinity is important  in the drug design pipeline since a strong interaction is the first step in finding a selective drug. However, prediction of the binding affinity value still remains a challenge \cite{ozturk2018deepdta}. In this thesis, we propose a graph-based model to predict the drug target binding affinities.
% Continue with the actual proposal